\documentclass[twocolumn, a4paper]{ltjsarticle}

\usepackage[dvipsnames]{xcolor} % 色の名前を拡張
\usepackage{tcolorbox} % 枠本体
\tcbuselibrary{raster,skins,breakable,theorems} % 呪文

\newenvironment{qbox}[1]{ % qbox環境を定義して使いやすく
    \begin{tcolorbox}[
        colframe = RoyalBlue,
        colback = RoyalBlue!10!White,
        title = {#1},
        fonttitle = \bfseries,
        breakable = true
    ]
}{
    \end{tcolorbox}
}
\newenvironment{warnbox}[1]{ % warnbox環境を新しく定義する
    \begin{tcolorbox}[
        colframe = Yellow!40!Orange,
        colback = Yellow!20!White,
        title = {#1},
        fonttitle = \color{Black} \bfseries,
        breakable = true
    ]
}{
    \end{tcolorbox}
}

\begin{document}

\section{たしざん}
たしざんを示す。
\begin{qbox}{たしざん}
    たしざんをする。
    \begin{equation*}
        \Biggl\lbrace
        \begin{alignedat}{2}
            1+1 &= 2 \\
            2 &= 1+1
        \end{alignedat}
    \end{equation*}

\end{qbox}

\begin{warnbox}{だめだよ}
    これはまちがい。
    \begin{alignat*}{1}
        1+1 &= 0 \\
        2 &= 2+1
    \end{alignat*}
\end{warnbox}

\end{document}
