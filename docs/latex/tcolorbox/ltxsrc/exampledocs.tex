\documentclass[twocolumn, a4paper]{ltjsarticle}

\usepackage[dvipsnames]{xcolor} % 色の名前を拡張
\usepackage{tcolorbox} % 枠本体
\tcbuselibrary{raster,skins,breakable,theorems} % 呪文
\usepackage{enumerate} % 箇条書き拡張

\newenvironment{qbox}[1]{ % qbox環境を定義して使いやすく
    \begin{tcolorbox}[
        colframe = RoyalBlue,
        colback = RoyalBlue!10!White,
        title = {#1},
        fonttitle = \bfseries,
        breakable = true
    ]
}{
    \end{tcolorbox}
}
\newenvironment{warnbox}[1]{ % warnbox環境を新しく定義する
    \begin{tcolorbox}[
        colframe = Yellow!40!Orange,
        colback = Yellow!20!White,
        title = {#1},
        fonttitle = \color{Black} \bfseries,
        breakable = true
    ]
}{
    \end{tcolorbox}
}
\newenvironment{exambox}[1]{ % exambox環境を新しく定義する
    \begin{tcolorbox}[
        colframe = OliveGreen,
        colback = ForestGreen!20!White,
        title = {#1},
        fonttitle = \bfseries,
        breakable = true
    ]
}{
    \end{tcolorbox}
}
\newenvironment{greaterbox}[1]{ % greaterbox環境を新しく定義する
    \begin{tcolorbox}[
        colframe = Maroon,
        colback = Dandelion!20!Brown!10!White,
        title = {#1},
        fonttitle = \color{White} \bfseries,
        breakable = true
    ]
}{
    \end{tcolorbox}
}

\setlength{\columnseprule}{0.4pt} % 中央の線

\begin{document}
\title{たしざんワーク}
\author{jnc-explosion}
\date{}

\twocolumn[
    \vspace{-10em}
    \maketitle

    \begin{abstract}
        \LaTeX は、いいよ。
    \end{abstract}

    \hrulefill
]

\section{たしざん}
たしざんを示す。
\begin{qbox}{たしざん}
    たしざんをする。
    \begin{equation*}
        \Biggl\lbrace
        \begin{alignedat}{2}
            1+1 &= 2 \\
            2 &= 1+1
        \end{alignedat}
    \end{equation*}

\end{qbox}

\begin{warnbox}{だめだよ}
    これはまちがい。
    \begin{alignat*}{1}
        1+1 &= 0 \\
        2 &= 2+1
    \end{alignat*}
\end{warnbox}

\begin{exambox}{やってみよう}
    これをつかって、もんだいをといてみよう。
    \tcblower
    \begin{enumerate}[(1).]
        \item $1+1=$
        \item $2+1=$
        \item $3+1=$
        \item $4+1=$
        \item 
        \begin{equation*}
            \mbox{\hspace{-45px}}10\times\lim_{n \to \infty}{\frac{1}{n} \sum_{k=1}^{n}{f\left(\frac{k}{n}\right)}}+1
        \end{equation*}
        \item $6+1$
        \item $7+1$
        \item $8+1$
        \item $9+1$
    \end{enumerate}
\end{exambox}

\newpage

\section{たしざんのおうよう}

\begin{greaterbox}{より良いたしざんのために}
    たしざんは、昔から人々を支えている。

    古代文明では既に開発されていて、計量・測量に利用されてきた。
    さんすう(X\hspace{-1.2pt}V\hspace{-1.2pt}I\hspace{-1.2pt}I\hspace{-1.2pt}I)の基礎にはたしざんが存在する。しかし、人々はたしざんについて特になにも思っていない。

    ところで、計算凡間違省によると、自分の9割5分はたしざん間違いで応用問題を落としている。
    また、その時にTwitter上で「たしざんできません」という呟きも確認されている。

    これは、たしざんの無意識化によって「たしざんの風化」が始まっていることの大きな証拠である。

    「たしざんの風化」は、我々の知らない内に、自分にのしかかっているのだ。しかも、この問題に気づいているテスト中の自分は、まだ少ない。

    我々は、よりよい「たしざん」のために、何に取り組むべきだろうか。
    \tcblower
    \begin{enumerate}[Q1: ]
        \item 「たしざんの風化」とは何でしょうか。
        \item あなたができる、よりよい「たしざん」のための取り組みを答えなさい。
        \item 作者は暇なのでしょうか。
        \item たしざんは必ず失敗します。その理由を答えなさい。
    \end{enumerate}
\end{greaterbox}

\end{document}
